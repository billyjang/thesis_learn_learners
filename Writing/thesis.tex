\documentclass [11pt] {report}

\def\postscript#1{\special{#1}}

\newtheorem{theorem}{Theorem}[chapter]
\newtheorem{lemma}[theorem]{Lemma}
\newtheorem{corollary}[theorem]{Corollary}
\newtheorem{conjecture}[theorem]{Conjecture}
\newtheorem{fact}[theorem]{Fact}

\def\sl{\it}
\def\csc{\it}
\def\focs#1{Proceedings of the #1 IEEE Symposium on the Foundations of
Computer Science}
\def\stoc#1{Proceedings of the #1 ACM Symposium on Theory of Computing}

\def\th{^{th}}
\def\st{^{st}}

\def\bb {\hbox{			% A box that I use at the end of proofs.
    \vrule height 8pt width .4pt\hskip -.4pt
    \vbox{\hrule width 8pt height .4pt\vskip -.4pt
          \vskip 8pt
	  \vskip -.4pt\hrule width 8pt height .4pt}
    \hskip -3.9pt
    \vrule height 8pt width .4pt}\par\medskip}

\def\proof{\noindent{\it Proof.}\ }

\evensidemargin 0in
\oddsidemargin 0in
\textwidth 6.5in
\topmargin 0in
\textheight 8in

\begin{document}
\baselineskip 21pt
\pagenumbering{roman}
\evensidemargin 0.25in
\footnotesep 14pt

\begin{titlepage}
\mbox{}
\vspace{1in}
\begin{center}
\LARGE An Example of Thesis Formatting \\
with \LaTeX \\[2in]
\normalsize Lyle Andrew McGeoch \\[\medskipamount]
March 27, 1990 \\[1in]
Submitted to the \\
Department of Mathematics and Computer Science \\
of Amherst College \\
in partial fulfillment of the requirements \\
for the degree of \\
Bachelor of Arts with Distinction
\end{center}
\vfill
\begin{center}
Copyright \copyright\ 1990 Lyle Andrew McGeoch
\end{center}
\end{titlepage}

\evensidemargin 0in

\chapter*{Abstract}
This short example thesis meets all of the formatting requirements in
Mathematics and Computer Science.

\chapter*{Acknowledgements}

\mbox{}\indent Thanks, Mom!

\tableofcontents
% \listoffigures
\newpage

\pagenumbering{arabic}

\chapter{Introduction}

The study of {\sl graph invariants}, properties of graphs that are preserved
by isomorphism, is central to graph theory.  Such properties are independent
of the representation or labelling of graphs.  There are many examples of
interesting graph invariants, such as vertex-count, edge-count, chromatic
number, crossing-number, maximum clique-size, and characteristic equation.
However, none of these invariants is a {\sl complete invariant} that is
sufficient to distinguish non-isomorphic graphs.  There is no known
complete invariant that can be determined in polynomial-time; the existence
of such an invariant would imply a polynomial-time test for graph
isomorphism.

Topological graph theory, which examines how graphs can be
imbedded in surfaces of different genus, may
provide the insights needed to obtain a complete invariant that can be
computed quickly.  This thesis presents results on two topological
invariants:  the {\sl maximum genus} and the {\sl genus distribution
}.

A graph has cellular imbeddings in surfaces of each genus between some
minimum and some maximum.  Progress has been made on understanding both
minimum- and maximum-genus imbeddings.  There is a polynomial-time algorithm
to find imbeddings in surfaces of fixed genus \cite{fmr}, and maximum genus
has been related to other graph invariants \cite{xuong79a}.  Chapter
2 presents a new polynomial-time algorithm that computes the
maximum genus of a graph and finds a maximum-genus imbedding.  This seems to
be the first polynomial-time algorithm that computes an interesting
imbedding and runs in time independent of the genus of the imbedding
surface.  The maximum-genus imbedding algorithm will also appear in
\cite{fgm}.

Chapter 3 presents results on the genus distribution of a graph,
an invariant that describes how many cellular imbeddings there are in
surfaces of different genus.  We completely calculate this distribution for
two classes of graphs, {\sl circular} and {\sl M\"obius ladders} and
consider it as a tool for determining graph isomorphism.  Genus distribution
is not a complete invariant: there are an infinite number of pairs of
non-isomorphic graphs with the same genus distribution. Experimental
evidence also suggests that genus distribution is not a sufficiently strong
invariant to be useful in testing the isomorphism of similar {\sl
strongly-regular} graphs.

\section{Preliminaries about Graph Imbeddings}

In topological graph theory, a {\sl graph} is defined to be a (possibly)
non-simplicial 1-complex.  In other words, multiple adjacencies and
self-loops are permitted.  In this work, we consider only simplicial
(simple) graphs.  Any graph containing self-loops and multiple adjacencies
can be transformed into a simplicial graph by inserting one or more vertices
in the interior of these edges.  Moreover, the resulting graph is
homeomorphic to the original graph.  This enables us to simplify the
notation.  We use the standard definitions relating to graphs (see, for
example, Harary \cite{harary}).  All the graphs we discuss will be connected
and undirected.

\subsection{Surfaces}  Our terminology is compatible with that of Gross and
Tucker \cite{gt} and of White \cite{white}.

The {\it topological spaces} of interest here are all homeomorphic to
subspaces of $E^3$.  A {\it homeomorphism} between two topological spaces is
a continuous bijective mapping with a continuous inverse.  A connected
two-dimensional subspace is a {\it surface} if every point has a neighborhood that
is homeomorphic to the closed unit disk.  A surface $S$ is {\it orientable}
if it does not contain a M\"obius band.

We deal only with compact orientable surfaces.  Every such surface $S$ is
homeomorphic to a generalized torus.  The number of handles is denoted
$\gamma(S)$ and is called the {\it genus} of the surface.  A sphere, for
example, is a surface of genus $0$, a torus is a surface of genus $1$, and a
$2$-handled torus is a surface of genus $2$.

\subsection {Graph imbeddings and faces} 
Although a graph is an abstract
combinatorial object, there is a topological representation of it:  in
Euclidean 3-space, we represent each vertex by a distinct point and each
edge by a distinct curve between the two endpoints, where a {\it curve}
means a homeomorphic image of the unit interval [0,1].  We require that the
interior of an edge intersect no other edge or vertex of the graph.  When
referring to a graph in a topological setting, we mean such a
representation.

An {\it imbedding} $G \rightarrow S$ of a graph $G$ in the surface $S$ is a
continuous one-to-one mapping.  The components of $S-G$ are
called {\it regions}.  If each region is homeomorphic to an open disk, the
imbedding is {\it cellular}, and the regions are called {\it faces}.  All our
imbeddings are cellular.  The set of faces of an imbedding is denoted $F$.
The Euler polyhedral equation 
\begin{equation}
|V| - |E| + |F| = 2 - 2\gamma(S)
\label{eulereqn}
\end{equation} holds for all cellular imbeddings.  

\addcontentsline{toc}{chapter}{\protect\numberline{}{Bibliography}}
\bibliographystyle{plain}
\bibliography{thesis}

\end{document}
